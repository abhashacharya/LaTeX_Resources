\documentclass{beamer}
\usetheme{CambridgeUS}
\title{Fundamentals of \LaTeX}
%\subtitle{Using Beamer}
\author{Abhash Acharya}
\date {May 31-June 3, 2023}
\begin{document}
\maketitle
%\begin{frame}
%\frametitle{Table of Contents}
%\tableofcontents
%\end{frame}
\begin{frame}
\frametitle{Introduction to \LaTeX}
\section{Introduction to \LaTeX}
\begin{itemize}
\item History of \LaTeX begins with a program called \TeX. \pause
\item In 1978, a computer scientist Donal Knuth developed \TeX. \pause
\item \LaTeX was originally written by Leslie Lamport \pause and is based on the \TeX typesetting engine by Donald Knuth. \pause
\item \LaTeX is a free, open-source software for typesetting documents. \pause
\item \LaTeX uses style files extensively called classes and packages making it easy to design and to modify the appearance of the whole document and all of its details. \pause
\item not WYSIWYG
\end{itemize}
\end{frame}
\begin{frame}
\frametitle{Fundamental Components of \LaTeX}
\section{Fundamental Components of \LaTeX}
\begin{itemize}
\item documentclass \pause
%\begin{verbatim}
%\documentclass [optional argument]{argument}
%\end{verbatim} 
\item begin-document \pause
%\begin{verbatim}
%\begin{document}
%\end{verbatim} 
\item end-document 
%\begin{verbatim}
%\end{document}
%\end{verbatim}
\end{itemize}
\pause
\begin{block}{Remark}
This is our first presentation using Beamer.
\end{block}
\pause
\begin{alertblock} {Alert}
Writing codes might be boring.
\end{alertblock}
\pause
\begin{examples} {documentclass}
The document class may be book, article, report, letter or beamer.
\end{examples}
\end{frame}
\end{document}